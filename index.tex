% Options for packages loaded elsewhere
\PassOptionsToPackage{unicode}{hyperref}
\PassOptionsToPackage{hyphens}{url}
%
\documentclass[
]{article}
\title{MyLabJournal}
\author{}
\date{\vspace{-2.5em}}

\usepackage{amsmath,amssymb}
\usepackage{lmodern}
\usepackage{iftex}
\ifPDFTeX
  \usepackage[T1]{fontenc}
  \usepackage[utf8]{inputenc}
  \usepackage{textcomp} % provide euro and other symbols
\else % if luatex or xetex
  \usepackage{unicode-math}
  \defaultfontfeatures{Scale=MatchLowercase}
  \defaultfontfeatures[\rmfamily]{Ligatures=TeX,Scale=1}
\fi
% Use upquote if available, for straight quotes in verbatim environments
\IfFileExists{upquote.sty}{\usepackage{upquote}}{}
\IfFileExists{microtype.sty}{% use microtype if available
  \usepackage[]{microtype}
  \UseMicrotypeSet[protrusion]{basicmath} % disable protrusion for tt fonts
}{}
\makeatletter
\@ifundefined{KOMAClassName}{% if non-KOMA class
  \IfFileExists{parskip.sty}{%
    \usepackage{parskip}
  }{% else
    \setlength{\parindent}{0pt}
    \setlength{\parskip}{6pt plus 2pt minus 1pt}}
}{% if KOMA class
  \KOMAoptions{parskip=half}}
\makeatother
\usepackage{xcolor}
\IfFileExists{xurl.sty}{\usepackage{xurl}}{} % add URL line breaks if available
\IfFileExists{bookmark.sty}{\usepackage{bookmark}}{\usepackage{hyperref}}
\hypersetup{
  pdftitle={MyLabJournal},
  hidelinks,
  pdfcreator={LaTeX via pandoc}}
\urlstyle{same} % disable monospaced font for URLs
\usepackage[margin=1in]{geometry}
\usepackage{graphicx}
\makeatletter
\def\maxwidth{\ifdim\Gin@nat@width>\linewidth\linewidth\else\Gin@nat@width\fi}
\def\maxheight{\ifdim\Gin@nat@height>\textheight\textheight\else\Gin@nat@height\fi}
\makeatother
% Scale images if necessary, so that they will not overflow the page
% margins by default, and it is still possible to overwrite the defaults
% using explicit options in \includegraphics[width, height, ...]{}
\setkeys{Gin}{width=\maxwidth,height=\maxheight,keepaspectratio}
% Set default figure placement to htbp
\makeatletter
\def\fps@figure{htbp}
\makeatother
\setlength{\emergencystretch}{3em} % prevent overfull lines
\providecommand{\tightlist}{%
  \setlength{\itemsep}{0pt}\setlength{\parskip}{0pt}}
\setcounter{secnumdepth}{-\maxdimen} % remove section numbering
\ifLuaTeX
  \usepackage{selnolig}  % disable illegal ligatures
\fi

\begin{document}
\maketitle

\hypertarget{my-lab-journal}{%
\section{\texorpdfstring{\textbf{My Lab
Journal}}{My Lab Journal}}\label{my-lab-journal}}

This can be a template for Scripts containing Content and Code for
Classes of the Business Analyrics and Information Systems Chair at JMU
Würzburg.

\hypertarget{how-to-use}{%
\subsection{How to use}\label{how-to-use}}

\begin{enumerate}
\def\labelenumi{\arabic{enumi}.}
\item
  fork the repo for this website and follow instructions on read me to
  get set up. \url{https://github.com/CrumpLab/LabJournalWebsite}
\item
  Blog/journal what you are doing in R, by editing the Journal.Rmd. See
  the
  \href{https://crumplab.github.io/LabJournalWebsite/Journal.html}{Journal
  page} for an example of what to do to get started learning R.
\item
  See the
  \href{https://crumplab.github.io/LabJournalWebsite/Links.html}{links
  page} for lots of helpful links on learning R.
\item
  Change everything to make it your own.
\end{enumerate}

\hypertarget{crump-lab-human-cognition-and-performance}{%
\subsubsection{\texorpdfstring{\href{https://crumplab.github.io}{Crump
Lab: Human Cognition and
Performance}}{Crump Lab: Human Cognition and Performance}}\label{crump-lab-human-cognition-and-performance}}

\includegraphics{images/logo.png}

\end{document}
